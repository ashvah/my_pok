%
%                               POK header
%
% The following file is a part of the POK project. Any modification should
% be made according to the POK licence. You CANNOT use this file or a part
% of a file for your own project.
%
% For more information on the POK licence, please see our LICENCE FILE
%
% Please follow the coding guidelines described in doc/CODING_GUIDELINES
%
%                                      Copyright (c) 2007-2022 POK team

\chapter{Configuration directives}

This chapter details the different configuration directives for kernel and
partitions. The configuration of kernel and partitions is made using C code. You
must write it carefully since a mistake can have significant impacts in terms of
safety or security.

Most of the time, the C configuration code will be macros in global variables.
The purpose of this chapter is to detail each variable. If you use generated
code, the configuration code is mostly generated in \texttt{deployment.c} and
\texttt{deployment.h} files.

\section{Automatic configuration from ARINC653 XML files}
\label{section-arinc653-xml-configuration}
You can automatically generate the configuration of your kernel using ARINC653
XML deployment files. For that, we designed a tool that analyzes ARINC653 XML
files and automatically produce the C configuration code (\texttt{deployment.h}
and \texttt{deployment.c}).

However, the configuration produced is not as complete as the one generated from
AADL models. Indeed, ARINC653 XML files do not contain enough information to
generate the whole configuration and is not sufficient to generate the
configuration of partitions. However, this is a good way to have basic
configuration files that can be improved by manual editing.

The tool is located in the \texttt{misc} directory of POK releases. You can use
it as it:
\begin{verbatim}
misc/arinc653-xml-conf.pl arinc653-configuration-file.xml
\end{verbatim}

When it is invoked, this program automatically produces two files:
\texttt{deployment.h} and \texttt{deployment.c}. These files must be compiled
with the kernel for its automatic configuration.


\section{Common configuration}
The following macros can be defined for both partitions and kernel:
\begin{itemize}
   \item
      \texttt{POK\_GENERATED\_CODE}: specify that the code compiled has been
      generated from AADL so that we can restrict and avoid the use of some
      functions. This maccro is automatically added by \ocarina when it generates
      code from \aadl models.
\end{itemize}

\section{Kernel configuration}
   \subsection{Services activation}
   You can define which capabilities you want in the kernel by defining
   some macros. Depending on which maccro you define, it will add services and
   capabilities in your kernel. It was made to make a very tight kernel and
   ease verification/certification efforts.

   When you use code generation capabilities, these declarations are
   automatically created in the \texttt{deployment.h} file.

   \begin{itemize}
      \item
         \texttt{POK\_NEEDS\_PARTITIONS} maccro indicates that you need partitioning
         services. It implies that you define configuration macros and
         variables for the partitioning service.
      \item
         \texttt{POK\_NEEDS\_SCHED} maccro specifies that you need the
         scheduler.
      \item
         \texttt{POK\_NEEDS\_PCI} maccro specifies that kernel will include
         services to use PCI devices.
      \item
         \texttt{POK\_NEEDS\_IO} maccro specifies that input/output service must
         be activated so that some partitions will be allowed to perform i/o.
      \item
         \texttt{POK\_NEEDS\_DEBUG} maccro specifies that debugging information
         are activated. Additional output will be produced.
      \item
         \texttt{POK\_NEEDS\_LOCKOBJECTS} maccro specifies that you need the
         lockobject service. It must be defined if you use mutexes or
         semaphores.
      \item
         \texttt{POK\_NEEDS\_THREADS} maccro that thread service must be
         activated.
      \item
         \texttt{POK\_NEEDS\_GETTICK} maccro that time service must be activated
         (interrupt frame on timer interrupt is installed and clock is available).
      \item
         \texttt{POK\_NEEDS\_SCHED\_RR} : the Round Robin scheduling policy is
         included in the kernel.
      \item
         \texttt{POK\_NEEDS\_SCHED\_RMS} : the Rate Monotonic Scheduling
         scheduling policy is included in the kernel.
      \item
         \texttt{POK\_NEEDS\_SCHED\_EDF} : the Earliest Deadline First
         scheduling policy is included in the kernel.
      \item
         \texttt{POK\_NEEDS\_SCHED\_LLF} : the Last Laxity First scheduling
         protocol is included in the kernel.
      \item
         \texttt{POK\_NEEDS\_SCHED\_STATIC} : the static scheduling protocol is
         included in the kernel.
      \item
         \texttt{POK\_NEEDS\_PORTS\_SAMPLING} : the sampling ports service for
         inter-partitions communication is included.
      \item
         \texttt{POK\_NEEDS\_PORTS\_QUEUEING} : the queueing ports service for
         inter-partitions communication is included.
   \end{itemize}

   \subsection{General configuration}
      \subsubsection{Number of threads}
      The \texttt{POK\_CONFIG\_NB\_THREADS} maccro specifies the number of
      threads in the system. This represents how many threads can be handled in
      the kernel.

      The values must be computed like this : number of threads in your system 
      + 2. In fact, in this maccro, you must add 2 additional threads : the
      kernel thread and the idle thread.

      \subsubsection{Number of lockobjects}
      The \texttt{POK\_CONFIG\_NB\_LOCKOBJECTS} maccro specifies the number of
      lockobjects the kernel would manage. It is the sum of potential
      semaphores, mutexes or ARINC653 events.


   \subsection{Partitions configuration}

      \subsection{Number of partitions}
      The \texttt{POK\_CONFIG\_NB\_PARTITIONS} maccro specifies the number of
      partitions handled in the kernel.


      \subsubsection{Threads allocation across partitions}
      The \texttt{POK\_CONFIG\_PARTITIONS\_NTHREADS} maccro specifies how many
      threads would resides in partitions. This declaration is an array, each
      value of the array corresponds to the number of threads inside a
      partition.

      An example is given below. In this example, we consider that we have 4
      partitions. The first, second and third partitions handle two threads while
      the last partition has 4 threads.

      \subsubsection{Number of nodes}
      The \texttt{POK\_CONFIG\_NB\_NODES} specifies the number of nodes in the
      distributed system if you use a such architecture. It is useful if you
      have more than one node and use network capabilities.

\begin{verbatim}
      #define POK_CONFIG_PARTITIONS_NTHREADS {2,2,2,4}
\end{verbatim}


      \subsubsection{Lockobjects allocation across partitions}
      The \texttt{POK\_CONFIG\_PARTITIONS\_NLOCKOBJECTS} specifies the number of
      lock objects for each partition. This declaration is an array, each value
      $n$ specifies how many lock objects we have for partition $n$.

      There is an example of the use of this configuration directive. Here, the
      first partition will have one lockobject while the second partition will
      have three lockobjects.

\begin{verbatim}
#define POK_CONFIG_PARTITIONS_NLOCKOBJECTS {1,3}
\end{verbatim}


      \subsubsection{Scheduler of each partition (level 1 of scheduling)}
      The \texttt{POK\_CONFIG\_PARTITIONS\_SCHEDULER} specifies the scheduler
      used in each partition. This declaration is an array, each value $n$
      corresponds to the scheduler used for partition $n$.

      There is an example below. Here, the four partitions used the
      \textit{Round-Robin} scheduler.

\begin{verbatim}
#define POK_CONFIG_PARTITIONS_SCHEDULER {POK_SCHED_RR,POK_SCHED_RR,POK_SCHED_RR,POK_SCHED_RR}
\end{verbatim}


      \subsubsection{Scheduler of partitions (level 0 of scheduling)}
      The scheduling of partitions is specified with several macros.

      The \texttt{POK\_CONFIG\_SCHEDULING\_NBSLOTS} specifies the number of time
      frames allocated for partitions execution.

      The \texttt{POK\_CONFIG\_SCHEDULING\_SLOTS} specifies the size (in
      milliseconds) of each slot.

      The \texttt{POK\_CONFIG\_SCHEDULING\_SLOTS\_ALLOCATION} specified the
      allocation of each slot. In other words, which partition is scheduling at
      which slot. The declaration is an array and the value $n$ specifies which
      partition uses the slot $n$.

      The \texttt{POK\_CONFIG\_MAJOR\_FRAME} specifies the major frame, the time
      when inter-partitions ports are flushed. It corresponds to the end of a
      scheduling cycle.

      An example is provided below. Here, we have four partitions. We declare 4
      slots of 500ms. The first slot is for the first partition, the second slot
      for the second partition and so on. The major frame (time when scheduling
      slots are repeated) is 2s (2000ms).

\begin{verbatim}
#define POK_CONFIG_SCHEDULING_SLOTS {500,500,500,500}

#define POK_CONFIG_SCHEDULING_SLOTS_ALLOCATION {0,1,2,3}

#define POK_CONFIG_SCHEDULING_NBSLOTS 4

#define POK_CONFIG_SCHEDULING_MAJOR_FRAME 2000
\end{verbatim}


      \subsubsection{Partitions size}
      The \texttt{POK\_CONFIG\_PARTITIONS\_SIZE} maccro specifies an array with
      partitions size in bytes. The declaration is an array, each value $n$
      represent the size of partition $n$.

      There is an example of a such declaration below. Here, we have 4
      partitions. The three first partition have a size of 80000 bytes while the
      last one has a size of 85000 bytes.

\begin{verbatim}
#define POK_CONFIG_PARTITIONS_SIZE {80000,80000,80000,85000}
\end{verbatim}


   \subsection{Inter-partitions ports communication}
   For inter-partitions communication, we introduce several concepts:
   \begin{itemize}
      \item
         The \textbf{node identifier} is a unique number for each node.
      \item
         The \textbf{global port identifier} is a unique number for each port in
         the whole distributed system. This unique identifier identifies each
         port of each node.
      \item
         The \textbf{local port identifier} is a unique number for each port on
         the local node \textbf{only}. It identifies each inter-partition
         communication port on the local kernel.
   \end{itemize}

   So, for each node, you must specify in the kernel:
   \begin{itemize}
      \item
         The node identifier of the current node
      \item
         The number of nodes in the distributed system
      \item
         The number of inter-partitions ports in the distributed system
      \item
         The number of inter-partitions ports on the local node
      \item
         All identifiers of global ports
      \item
         All identifiers of local ports
      \item
         The association between global ports and nodes
      \item
         The association between global ports and local ports
      \item
         The association between local ports and global ports
   \end{itemize}

   \subsubsection{Current node identifier}
   The identifier of the current node is specified with the
   \texttt{POK\_CONFIG\_LOCAL\_NODE} maccro.

   When you use code generation, this declaration is
   automatically created in the \texttt{deployment.h} file.

   \subsubsection{Number of global ports}
   The number of global ports in the distributed system is specified
   with the \texttt{POK\_CONFIG\_NB\_GLOBAL\_PORTS}. It indicates the number of
   global ports in the system.

   When you use code generation, this declaration is
   automatically created in the \texttt{deployment.h} file.

   \subsubsection{Number of local ports}
   The number of local ports in the current node is specified
   using the \texttt{POK\_CONFIG\_NB\_PORTS} maccro. It specifies the number of
   ports on the local node.

   When you use code generation, this declaration is
   automatically created in the \texttt{deployment.h} file.

   \subsubsection{Local ports identifiers}
   The local ports identifiers are specified in an enum with the identifier
   \texttt{pok\_port\_local\_identifier\_t}. In this enum, you must ALWAYS
   add an identifier for an invalid identifier called
   \texttt{invalid\_identifier}.
   Note that this enum declaration specifies the local ports of the current node
   and consequently, it is dependent on each node communication requirements.

   When you use code generation, this declaration is
   automatically created in the \texttt{deployment.h} file.

   There is an example of a such enum
   declaration:
   \begin{verbatim}
   typedef enum
   {
     node1_partition_secret_outgoing = 0,
     node1_partition_topsecret_outgoing = 1,
     node1_partition_unclassified_outgoing = 2,
     invalid_local_port = 3
   } pok_port_local_identifier_t;
   \end{verbatim}

   \subsubsection{Global ports identifiers}
   The global ports identifiers is specified using an enum called
   \texttt{pok\_port\_identifier\_t}. This enum declaration must be \textbf{THE SAME}
   on all node of the distributed system.

   When you use code generation, this declaration is
   automatically created in the \texttt{deployment.h} file.

   There is an example of a such enum declaration:
   \begin{verbatim}
typedef enum
{
  node1_partition_secret_outgoing_global = 0,
  node1_partition_topsecret_outgoing_global = 1,
  node1_partition_unclassified_outgoing_global = 2,
  node2_partition_secret_incoming_global = 3,
  node2_partition_topsecret_incoming_global = 4,
  node2_partition_unclassified_incoming_global = 5
} pok_port_identifier_t;
   \end{verbatim}

   \subsubsection{Node identifiers}
   The node identifiers are specified by declaring the
   \texttt{pok\_node\_identifier\_t} type. It contains the value of each node
   identifier. Please also note that the \texttt{POK\_CONFIG\_LOCAL\_NODE} value
   must be in this enum declaration. This enum declaration is \textbf{THE SAME} on
   all nodes of the distributed system.

   When you use code generation, this declaration is
   automatically created in the \texttt{deployment.h} file.

   There is an example of a such declaration
   \begin{verbatim}
   typedef enum
   {
     node1 = 0,
     node2 = 1
   } pok_node_identifier_t;
   \end{verbatim}

   \subsubsection{Associate local ports with global ports}
   We specify the global port of each local port with the
   \texttt{pok\_ports\_identifiers} array. An example is given below:

   \begin{verbatim}
   uint8_t pok_ports_identifiers[POK_CONFIG_NB_PORTS] =
            {node1_partition_secret_outgoing,
            node1_partition_topsecret_outgoing,
            node1_partition_unclassified_outgoing};
   \end{verbatim}

   Here, the first local port of the current node corresponds to the
   \texttt{node1\_partition\_secret\_outgoing} global port. 

   When you use code generation, this declaration is
   automatically created in the \texttt{deployment.c} file.


   \subsubsection{Specify local ports routing (local ports to global ports)}
   For each local port, we specify the number of destinations. Since there can
   be more than one recipient to a sending port, we specify how many ports
   should receive data. We specify that with the
   \texttt{pok\_ports\_nb\_destinations} array.

   Then, we specify the local port routing policy with the
   \texttt{pok\_ports\_destinations} array. In this array, each value is a
   pointer to another array that contains the recipient global port values.

   An example is given below. Here, the first local port has one recipient. The
   recipient list is specified with the first elements of the
   \texttt{pok\_ports\_destinations} array, which is the
   \texttt{node1\_partition\_secret\_outgoing\_deployment\_destinations} array.
   Thus, we can see that the recipient port identifier is
   \texttt{node2\_partition\_secret\_incoming\_global}.

   \begin{verbatim}
   uint8_t node1_partition_secret_outgoing_deployment_destinations[1] =
      {node2_partition_secret_incoming_global};
   uint8_t node1_partition_secret_partport[1] =
      {node1_partition_secret_outgoing};
   uint8_t node1_partition_topsecret_outgoing_deployment_destinations[1] =
      {node2_partition_topsecret_incoming_global};
   uint8_t node1_partition_unclassified_outgoing_deployment_destinations[1] =
      {node2_partition_unclassified_incoming_global};

   uint8_t pok_ports_nb_destinations[POK_CONFIG_NB_PORTS] = {1,1,1};

   uint8_t* pok_ports_destinations[POK_CONFIG_NB_PORTS] =
      {node1_partition_secret_outgoing_deployment_destinations,
       node1_partition_topsecret_outgoing_deployment_destinations,
       node1_partition_unclassified_outgoing_deployment_destinations};
   \end{verbatim}


   \subsubsection{Convert local port to global ports}
   The association (conversion) between each local and global ports is given
   with the \texttt{pok\_local\_ports\_to\_global\_ports} array. For each local
   port identifier, we specify the associated global port value.

   An example is given below. Here, the first local port corresponds to the
   global port identifier \texttt{node1\_partition\_secret\_outgoing\_global}.

   \begin{verbatim}
   uint8_t pok_local_ports_to_global_ports[POK_CONFIG_NB_PORTS] =
      {node1_partition_secret_outgoing_global,
       node1_partition_topsecret_outgoing_global,
       node1_partition_unclassified_outgoing_global};
   \end{verbatim}

   When you use code generation, this declaration is
   automatically created in the \texttt{deployment.c} file.


   \subsubsection{Convert global port to local port}
   It is sometimes needed to convert a global port value to a local port. You
   can have this information with the \texttt{pok\_global\_ports\_to\_local\_ports}.
   
   The definition of this array is different on all nodes. It specifies the
   local port identifier on the current node with each global port. If the
   global port is not on the current node, we specify the \texttt{invalid\_port}
   value.

   An example is given below. We can see that the three last ports are not
   located on the current node.

   \begin{verbatim}
   uint8_t pok_global_ports_to_local_ports[POK_CONFIG_NB_GLOBAL_PORTS] =
   {node1_partition_secret_outgoing,
    node1_partition_topsecret_outgoing,
    node1_partition_unclassified_outgoing,
    invalid_local_port,
    invalid_local_port,
    invalid_local_port};
   \end{verbatim}

   When you use code generation, this declaration is
   automatically created in the \texttt{deployment.c} file.

   \subsubsection{Location of each global port}
   The location of each global port is specified with the
   \texttt{pok\_ports\_nodes} array. It indicates, for each port, the associated
   node identifier.

   In the following example, it shows that the three first global ports are
   located on the node 0 and the other on the node 1.

   \begin{verbatim}
   uint8_t pok_ports_nodes[POK_CONFIG_NB_GLOBAL_PORTS] =
      {0,0,0,1,1,1};
   \end{verbatim}

   When you use code generation, this declaration is
   automatically created in the \texttt{deployment.c} file.


   \subsubsection{Specify the port type}
   The kernel must know the kind of each port (queuing or sampling). We specify
   that requirement with the \texttt{pok\_ports\_kind} array. There is an
   example of a such declaration below.

   \begin{verbatim}
   pok_port_kind_t pok_ports_kind[POK_CONFIG_NB_PORTS] =
   {POK_PORT_KIND_SAMPLING,POK_PORT_KIND_SAMPLING,POK_PORT_KIND_SAMPLING};
   \end{verbatim}

   Here, the three local ports are sampling ports. You can have three kind of
   ports:
   \begin{enumerate}
      \item
         \textbf{Sampling ports} (\texttt{POK\_PORT\_KIND\_SAMPLING}) : stores data but does not queue them.
      \item
         \textbf{Queuing ports} (\texttt{POK\_PORT\_KIND\_QUEUEING}) : queues every new instance of the data.
      \item 
         \textbf{Virtual ports} (\texttt{POK\_PORT\_KIND\_VIRTUAL}) : this port is not stored in the kernel and this
         is a virtual port. This port belongs to another machine. We add it only
         to create the routing policy in the distributed network. You cannot
         write or read data on/from virtual ports, only get the port identifier
         associated with them.
   \end{enumerate}

   When you use code generation, this declaration is
   automatically created in the \texttt{deployment.c} file.


   \subsubsection{Specify ports names}
   When the developer calls ports instanciation, he can specify a port name. For
   that reason, the kernel must know the name associated with each port.

   This information is provided by the \texttt{pok\_ports\_names} declaration.
   It contains the name of each local port.

   There is an example of a such declaration.
   \begin{verbatim}
   char* pok_ports_names[POK_CONFIG_NB_PORTS] =
      {"node1_partition_secret_outgoing",
       "node1_partition_topsecret_outgoing",
       "node1_partition_unclassified_outgoing"};
   \end{verbatim}

   When you use code generation, this declaration is
   automatically created in the \texttt{deployment.c} file.

   \subsubsection{Specify port usage for each partition}
   The inter-partition ports are dedicated to some partitions. Consequently, we
   have to specify in the configuration code which partition is allowed to
   read/write which port. 

   We do that with two arrays : \texttt{pok\_ports\_nb\_ports\_by\_partition} and 
   \texttt{pok\_ports\_by\_partition}.

   The  \texttt{pok\_ports\_nb\_ports\_by\_partition} indicates for each
   partition, the number of ports allocated. In the same manner, the
   \texttt{pok\_ports\_by\_partition} indicate an array that contains the global
   ports identifiers allowed for this partition.

   An example is provided. In this example, we see that the first partition has
   one port and the identifier of this port is
   \texttt{node1\_partition\_secret\_outgoing}.

   \begin{verbatim}
   uint8_t node1_partition_secret_partport[1] =
      {node1_partition_secret_outgoing};

   uint8_t node1_partition_topsecret_partport[1] =
      {node1_partition_topsecret_outgoing};

   uint8_t node1_partition_unclassified_partport[1] =
      {node1_partition_unclassified_outgoing};


   uint8_t pok_ports_nb_ports_by_partition[POK_CONFIG_NB_PARTITIONS] =
      {1,1,1};

   uint8_t* pok_ports_by_partition[POK_CONFIG_NB_PARTITIONS] =
      {node1_partition_secret_partport,
       node1_partition_topsecret_partport,
       node1_partition_unclassified_partport};
   \end{verbatim}

   When you use code generation, this declaration is
   automatically created in the \texttt{deployment.c} file.

\section{Libpok (partition runtime)}
   \section{Configuration}
   You define the configuration policy by defining some C-style macros. There
   are the list of useful macros:
   \begin{itemize}
      \item
         \texttt{POK\_CONFIG\_NB\_THREADS}: specify the number of threads
         contained in the partition.
      \item
         \texttt{POK\_CONFIG\_NB\_BUFFERS}:
         Specify the number of buffers
         used in the libpok (intra-partition communication).
      \item
         \texttt{POK\_CONFIG\_NB\_SEMAPHORES}: Specify the number of semaphores
         used in the libpok (intra-partition communication).
      \item
         \texttt{POK\_CONFIG\_NB\_BLACKBOARDS}: Specify the number of blackboard
         we use for intra-partition communications.
      \item
         \texttt{POK\_CONFIG\_NB\_EVENTS}: Specify the number of events we use
         for intra-partition communications.
      \item
         \texttt{POK\_CONFIG\_ALLOCATOR\_NB\_SPACES}: Indicate the number of
         spaces we should reserve in the memory allocator. Since the memory
         allocator tries to reach determinism, the number of space is fixed. So,
         you have to specify how many spaces you want by defining this maccro.
      \item
         \texttt{POK\_CONFIG\_ALLOCATOR\_MEMORY\_SIZE}: Indicate which amount of
         memory must be reserved for the memory allocator.
      \item
         \texttt{POK\_HW\_ADDR}:
         Define the hardware address of the ethernet card. This maccro is useful
         if the partition implements a device driver for a network device. In
         POK and its libpok layer, we use it for the RTL8029 device driver.
   \end{itemize}

   \section{Services activation}
   To activate \textit{libpok} services, you must define some macros. By
   default, you don't have any services. You activate service by defining
   macros. Thus, it ensures that each partition contains only required services
   and avoid any memory overhead in partitions. 

   These macros have the form \texttt{POK\_NEEDS\_...}. There is a list of
   these macros:
   \begin{itemize}
      \item
         \texttt{POK\_NEEDS\_RTL8029}: activate the functions of the device
         driver that support the \textit{Realtek} 8029 ethernet card.
      \item
         \texttt{POK\_NEEDS\_STDLIB}: activate services of the standard
         library (everything you can find in
         \texttt{libpok/include/libc/stdlib.h}).
      \item
         \texttt{POK\_NEEDS\_STDIO}: activate the services of the standard
         Input/Output library (printf, etc.). You can find available functions
         in \texttt{libpok/include/libc/stdio.h}).
      \item
         \texttt{POK\_NEEDS\_IO}: needs functions to perform I/O. These
         functions are just system calls and ask the kernel to perform them. The
         partition \textbf{CANNOT} make any I/O by itself.
      \item
         \texttt{POK\_NEEDS\_TIME}: activate functions that handle time.
      \item
         \texttt{POK\_NEEDS\_THREADS}: activate functions relative to threads.
      \item
         \texttt{POK\_NEEDS\_PORTS\_VIRTUAL}: activate functions for virtual
         ports management. Virtual ports are handled by the kernel. So,
         activated functions in the libpok are just system call to the kernel to
         get the port routing policy. Since virtual ports represent ports that
         are located on other nodes, this maccro should be used only by
         partitions that actually implement network device drivers.
      \item
         \texttt{POK\_NEEDS\_PORTS\_SAMPLING}: activate interfacing functions with the
         kernel to use sampling ports.
      \item
         \texttt{POK\_NEEDS\_PORTS\_QUEUEING}: activate interfacing functions
         with the kernel to use queueing ports.
      \item
         \texttt{POK\_NEEDS\_ALLOCATOR} : activate the memory allocator of the
         partition. This service can be configured with
         \texttt{POK\_CONFIG\_ALLOCATOR...} macros.
      \item
         \texttt{POK\_NEEDS\_ARINC653\_PROCESS}: activate the process service of
         the ARINC653 layer.
      \item
         \texttt{POK\_NEEDS\_ARINC653\_BLACKBOARD}: activate the blackboard
         service of the ARINC653 layer
      \item
         \texttt{POK\_NEEDS\_ARINC653\_BUFFER}:
         activate the buffer service of the ARINC653 layer.
      \item
         \texttt{POK\_NEEDS\_ARINC653\_SEMAPHORE}:
         activate the semaphore service of the ARINC653 layer.
      \item
         \texttt{POK\_NEEDS\_ARINC653\_QUEUEING}:
         activate the queueing service of the ARINC653 layer.
      \item
         \texttt{POK\_NEEDS\_ARINC653\_SAMPLING}:
         activate the sampling ports service of the ARINC653 layer.
      \item
         \texttt{POK\_NEEDS\_ARINC653\_ERROR}:
         activate the error service of the ARINC653 layer (health monitoring
         functions)
      \item
         \texttt{POK\_NEEDS\_BLACKBOARDS}:
         activate the blackboard service of POK (intra-partition communication)
      \item
         \texttt{POK\_NEEDS\_SEMAPHORES}:
         activate the semaphore service of POK (intra-partition communication)
      \item
         \texttt{POK\_NEEDS\_BUFFERS}:
         activate the buffer service of POK (intra-partition communication)
      \item
         \texttt{POK\_NEEDS\_ERROR\_HANDLING}:
         activate the error handling service in POK.
      \item
         \texttt{POK\_NEEDS\_DEBUG}:
         activate debug mode.
      \item
         \texttt{POK\_NEEDS\_LIBMATH}:
         activate the libmath, functions that are available in regular service
         by passing the \texttt{-lm} flag to the compiler. See
         \texttt{libpok/include/libm.h} file for the list of functions.
   \end{itemize}
